
\quad
\addtocounter{page}{-1}
\chapter*{Abstract}
\thispagestyle{empty}
Die Heterojunction (HJT) Solarzellen verwenden ein neuartige Bauweise, welche eine höheren Wirkungsgrad im Vergleich zu herkömmlichen Solarzellen haben. Das Potenzial dieser Technologie ist enorm was zu einer vermehrten Forschung selbiger führt. Für eine erfolgreiche Produktion von HJT Solarzellen müssen Methoden zur Qualitätssicherung entwickelt werden. Dies ist das Ziel dieser Arbeit. Der Schwerpunkt liegt dabei auf der Charakterisierung von Produktionsdefekten, mittels Photolumineszenz Messungen, welche während der Produktion entstehen. \\
Bisher werden im Projekt Produktionsdefekte nur mit herkömmlichen Bildverarbeitungsverfahren und mit überwachten Lernverfahren innerhalb des Deep-Learnings analysiert. Diese können Defekte oftmals sehr genau charakterisieren, bedürfen aber einem Hohen Maß an Anpassung auf die entsprechenden Messdaten. In den vergangenen Monaten und Jahren haben Unüberwachte Lernverfahren des Deep-Learnings enorme Fortschritte in der Vorhersagegenauigkeit erzielt. \\
Innerhalb dieser Arbeit sollen als Teil einer genauen Charakterisierung die Defekte welche während der Prozessierung der Zellen auftreten segmentiert werden. Weiterhin wird untersucht ob diese Defektstrukturen, mittels Unüberwachten Lernmethoden, in der gleichen Güte erkannt werden können. Dabei werden vielversprechnede Deep-Learning Methoden eingesetzt, welche keine Datenpaar zwischen Bild und Annotationen benötigen. Für das Ziel der Segmentation der Defekte wird ein gepaarter Datensatz erstellt, welcher aus Messungen verschiedener Waferrezepten besteht. \\
Für eine detaillierte Analyse werden die implementierten Modelle sowohl quantitativ als auch qualitativ ausgewertet. Die Auswertung zeigt, dass Deep-Learning Modelle in der Lage sind, Defekte anhand von Photolumineszenz Messungen erfolgreich zu segmentierten und diese wertvolle Informationen für die Charakterisierung der Zellen enthalten. 
