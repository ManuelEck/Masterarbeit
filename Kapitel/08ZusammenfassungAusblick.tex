\chapter{Zusammenfassung und Ausblick}

Das Ziel dieser Arbeit ist es Defekte auf HJT-Wafern welche durch den Prozessierungs Prozess entstehen zu detektierten. Diese Defekte werden aktuell nicht charakterisiert. Diese Arbeit gibt einen Ansatz diese Defekte mittels Deep Learing Methoden zu detektieren, welche für eine umfangreiche Charakterisierung der Wafer sehr wichtig ist. Es wird weiter untersucht ob, die Detektion mittels Unüberwachten Lernmethoden erfolgt werden kann um eine noch schnellere und Kosten effektivere Charakterisierung der Wafer vorzunehmen. \\

Da für die Vorliegenden Photolumineszenz Daten keine Annotationen vorliegen, macht dies, das trainieren und evaluieren von Überwachten Modellen schwer. Aus diesen Grund wurde ein gepaarter Datensatz für das Training und die Evaluation dieser Messungen erstellt.\\

Es wurden drei verschiedene Deep Learning Modelle implementiert und evaluiert. Die Modelle lassen sich in Unüberwachte Lernverfahren und Überwachte Lernverfahren einteilen. Die Unüberwachten Modelle wurden dabei auf den gesamten Daten trainiert während die Überwachten Modelle lediglich auf einem Teil dieser Daten, welche die erstellten Bildannotationen enthalten, trainiert und evaluiert werden. Es wurden verschiedene Experimente mit diesen Modellen ausgeführt und im Zuge der Unüberwachten Modelle eine neue Verlustfunktion, welche auf den Eigenschaften der verwendeten Daten basiert, entwickelt. Die Modelle wurden mit verschiedenen Konfigurationen der Verlustfunktionen für die Optimierung ausgeführt.\\

In in dieser Arbeit implementierten Modelle sind in der Lage Defekte auf photlumineszenz Messungen zu detektieren und segmentierten. Das Überwachte Modell hatte Probleme mit der Ungleichgewichtung der Klassen, welche sich  über eine Gewichtung des Verlustes ausgleichen ließ. Die Unüberwachten Modelle hatten das Problem, dass kein ausreichender Gegenpol zu den positiven Korrelationen der Verlustfunktion gefunden wurden. Um diesem Problem entgegenzuwirken, wurde eine Verlustfunktion entwickelt, welche die Position des verwerteten Bildausschnitts mit beachtet. Sowohl die Überwachten als auch die Unüberwachten Modelle zeigen gute Ergebnisse. Allerdings in der Qualitativen Auswertung eine höhere Güte in den Ergebnissen der Überwachten Modelle zu sehen. \\

Basierend auf der Evaluation der Ergebnisse, hat diese Arbeit weiterhin großes Potenzial für Verbesserungen und Erweiterungen. Der Datensatz könnte um weitere Zelltypen und Prozessierte Chargen erweitert werden um eine breitere Anwendung auf verschiedene Zelltypen zu gewährleisten. Ebenso ist eine Erweiterung der Unüberwachten Modelle mittels Conditional Random Fields, für eine bessere räumliche Auflösung, denkbar. 